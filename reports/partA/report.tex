\documentclass[11pt]{article}

\usepackage{fancyhdr}
\usepackage[margin=1in,headheight=13.6pt]{geometry}


%header and footer
\pagestyle{fancy}
\lhead{COMP30024 Artificial Intelligence}
\chead{}
\rhead{Assignment Part A}
\lfoot{}
\cfoot{\thepage}
\rfoot{}
\renewcommand{\headrulewidth}{0.4pt}

\begin{document}
    \title{\textbf{Assignment Part A}}
    \author{Rohan Hitchcock and Patrick Randell}
    \date{}
    \maketitle
    \section{Introduction}\label{sec:introduction}
    Contained in this report are the details of our approach to Project Part A: Searching, the first assignment in the course
    COMP30024: Artificial Intelligence at the University of Melbourne 2020 Semester 1.
    The game rules and assignment specification
    can be found here \{\}.

    \section{Problem Formulation}\label{sec:problem-formulation}
    After trialling many different methods for formulating the problem, we found that the only way for a search
    algorithm to solve all possible scenarios was to have knowledge of the entire board.
    Thus, the problem was formulated as follows:
    \subsection{States}\label{subsec:states}
    A state was defined as a board configuration, meaning the positions of all black and white stacks on the board, and
    the heights of those stacks.
    Included in a state are the goals: The positions white stacks need to be in, in order to pass the goal test.
    These goals are generated from the black stack positions by finding the collective explosion radius of
    recursively adjacent groups of black stacks.
    It was important to have these goals included as an attribute of a State, as in order to solve certain scenarios,
    a group of black stacks (ie, a goal group) needed to be removed (exploded) so others can be reached.
    \subsection{Actions}\label{subsec:actions}
    An action is defined as any valid move of a white stack in the current state.
    This can include moving an entire stack, a subset of a stack, exploding a stack or merging stacks.
    \subsection{Path Costs}\label{subsec:path-costs}
    The cost of a path to a solution is 1 for each move made, including "BOOM" moves during the path, or at the end.
    \subsection{Goal Test}\label{subsec:goal-test}
    The goal test for this problem was to check whether a white stack was in the explosion radius of every group of
    recursively adjacent black stacks in that state.
    In other words, if there was a white stack in every goal group.
    From here, the remaining white stacks could simply explode ("BOOM") resulting in the real solution, which is when
    no black stacks remain on the board.
    Note that the original goal groups may not necessarily be the ones remaining in the final goal test, as some may
    have been removed in order to reach the solution.
    \section{Search Algorithm}\label{sec:search-algorithm}
    We decided to choose the A* search algorithm for this problem.
    We chose A* because it is both complete and optimal, and allowed us to carefully design a heuristic that would
    optimise the search for our needs and requirements.
    Due to time being the main constraint for this assignment, we were less concerned about the space
    required by A* to store all nodes.
    \subsection{Heuristic}\label{subsec:heuristic}
    The heuristic we used for a state was the minimum of the sum of the minimum distances from each white token to
    each goal group, plus the number of goal groups in that state.
    Adding the number of goal groups meant that the algorithm was rewarded for removing goal groups.
    We also added a condition where if the sum of the heights of the white stacks was less than the number of
    remaining goal groups, the cost would be extremely high.
    This was to avoid A* expanding nodes that would not lead to a solution.
    A visualisation of this is below.
    Due to those added features to the heuristic, it is not admissible.
    \section{Complications and Problem Features}\label{sec:complications-and-problem-features}
    There were several "classes" of problems that needed to be considered when solving this searching problem.
    \subsection{Intersections}\label{subsec:intersections}
    huh
    \subsection{Stacking}\label{subsec:stacking}
    \subsection{Pre-exploding}\label{subsec:pre-exploding}
    \subsection{Combinations}\label{subsec:combinations}


\end{document}
